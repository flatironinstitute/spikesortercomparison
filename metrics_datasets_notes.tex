\documentclass[10pt]{article}
\textwidth 6.5in
\oddsidemargin=0in
\evensidemargin=0in
\usepackage{graphicx,bm,amssymb,amsmath,amsthm}
\newcommand{\bi}{\begin{itemize}}
\newcommand{\ei}{\end{itemize}}
\newcommand{\ben}{\begin{enumerate}}
\newcommand{\een}{\end{enumerate}}
\newcommand{\be}{\begin{equation}}
\newcommand{\ee}{\end{equation}}
\newcommand{\bea}{\begin{eqnarray}} 
\newcommand{\eea}{\end{eqnarray}}
\newcommand{\ba}{\begin{align}} 
\newcommand{\ea}{\end{align}}
\newcommand{\bse}{\begin{subequations}} 
\newcommand{\ese}{\end{subequations}}
\newcommand{\bc}{\begin{center}}
\newcommand{\ec}{\end{center}}
\newcommand{\bfi}{\begin{figure}}
\newcommand{\efi}{\end{figure}}
\newcommand{\ca}[2]{\caption{#1 \label{#2}}}
\newcommand{\ig}[2]{\includegraphics[#1]{#2}}
\newcommand{\tbox}[1]{{\mbox{\tiny #1}}}
\newcommand{\mbf}[1]{{\mathbf #1}}
\newcommand{\pO}{{\partial\Omega}}
\newcommand{\half}{\mbox{\small $\frac{1}{2}$}}
\newcommand{\vt}[2]{\left[\begin{array}{r}#1\\#2\end{array}\right]} % 2-col-vec
\newcommand{\mt}[4]{\left[\begin{array}{rr}#1&#2\\#3&#4\end{array}\right]} % 2x2
\newcommand{\xx}{\mbf{x}}
\newcommand{\ff}{\mbf{f}}
\newcommand{\yy}{\mbf{y}}
\newcommand{\RR}{\mathbb{R}}
\newcommand{\ttt}{{\bm \theta}}
\newcommand{\eps}{\varepsilon}
\newcommand{\eeps}{\bm{\eps}}
\newcommand{\HH}{{\cal H}}
\DeclareMathOperator{\re}{Re}
\DeclareMathOperator{\im}{Im}
\newtheorem{thm}{Theorem}
\newtheorem{cnj}[thm]{Conjecture}
\newtheorem{lem}[thm]{Lemma}
\newtheorem{cor}[thm]{Corollary}
\newtheorem{pro}[thm]{Proposition}
\newtheorem{rmk}[thm]{Remark}

\begin{document}
\title{Notes on metrics and datasets for community spike sorting validation}
\author{Alex Barnett}
\date{\today}
\maketitle
\begin{abstract}
  We sketch some ideas coming from the discussion in the last part of the
  Janelia spike sorting meeting of 2/22/18.
\end{abstract}

We are in the process of hosting at Flatiron a web-based platform where
a variety of spike-sorting packages are run on
a set of community-approved ground-truthed datasets, and their
performance metrics made publicly available online via an interactive
front-end (possibly also with an API).

Curating and gathering such datasets and metrics is a community effort.
Please give feedback and/or additions to this document.
As part of a


A meta-goal is to gather information about which quality metrics that
are computable without ground-truth most closely match the ground-truth
accuracy metric.

\section{Datasets}


At the meeting was discussed the following datasets with some form of
ground-truth. They fall into three categories.
\ben
\item Recordings with ground-truth
  \item

    
    
  \een

  It was noted that there are no ground-truthed recordings in
  awake, let alone behaving, animals.
  Please correct me if I got this wrong.

  
\section{Metrics}

It seems there is a useful classification of metrics, in descending order of
confidence.



\bi
\item Class I metrics.


\subsection{Other metrics}
