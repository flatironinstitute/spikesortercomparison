\documentclass[10pt]{article}
\textwidth 6.3in
\oddsidemargin=0in
\evensidemargin=0in
\usepackage{graphicx,bm,amssymb,amsmath,amsthm,url}

\newcommand{\bi}{\begin{itemize}}
\newcommand{\ei}{\end{itemize}}
\newcommand{\ben}{\begin{enumerate}}
\newcommand{\een}{\end{enumerate}}
\newcommand{\be}{\begin{equation}}
\newcommand{\ee}{\end{equation}}
\newcommand{\bea}{\begin{eqnarray}} 
\newcommand{\eea}{\end{eqnarray}}
\newcommand{\ba}{\begin{align}} 
\newcommand{\ea}{\end{align}}
\newcommand{\bse}{\begin{subequations}} 
\newcommand{\ese}{\end{subequations}}
\newcommand{\bc}{\begin{center}}
\newcommand{\ec}{\end{center}}
\newcommand{\bfi}{\begin{figure}}
\newcommand{\efi}{\end{figure}}
\newcommand{\ca}[2]{\caption{#1 \label{#2}}}
\newcommand{\ig}[2]{\includegraphics[#1]{#2}}
\newcommand{\tbox}[1]{{\mbox{\tiny #1}}}
\newcommand{\mbf}[1]{{\mathbf #1}}
\newcommand{\pO}{{\partial\Omega}}
\newcommand{\half}{\mbox{\small $\frac{1}{2}$}}
\newcommand{\vt}[2]{\left[\begin{array}{r}#1\\#2\end{array}\right]} % 2-col-vec
\newcommand{\mt}[4]{\left[\begin{array}{rr}#1&#2\\#3&#4\end{array}\right]} % 2x2
\newcommand{\xx}{\mbf{x}}
\newcommand{\ff}{\mbf{f}}
\newcommand{\yy}{\mbf{y}}
\newcommand{\RR}{\mathbb{R}}
\newcommand{\ttt}{{\bm \theta}}
\newcommand{\eps}{\varepsilon}
\newcommand{\eeps}{\bm{\eps}}
\newcommand{\HH}{{\cal H}}
\DeclareMathOperator{\re}{Re}
\DeclareMathOperator{\im}{Im}
\newtheorem{thm}{Theorem}
\newtheorem{cnj}[thm]{Conjecture}
\newtheorem{lem}[thm]{Lemma}
\newtheorem{cor}[thm]{Corollary}
\newtheorem{pro}[thm]{Proposition}
\newtheorem{rmk}[thm]{Remark}

\newcommand{\com}[1]{{\marginpar{\sffamily{\scriptsize #1}}}}
%\newcommand{\ahb}[1]{{\com{AB:#1}}}          % co-author comments


\begin{document}
\title{Proposal for web-based spike sorting validation}
\author{Alex Barnett, Jeremy Magland, and James Jun}
\date{Flatiron Institute, \today}
\maketitle
\begin{abstract}
  We lay out some choices for metrics and datasets for a community effort to
  compare and validate the major spike sorting packages, hosted at Flatiron.
  This includes summarizing and extending some ideas from the discussion at the
  Janelia spike sorting meeting of 3/22/18.
  This document will be open to suggestions from the electrophysiology community.
\end{abstract}

\section{Background and goals}

We are in the process of hosting at Flatiron a web-based platform where
a variety of spike-sorting packages are run on
a set of community-approved ground-truthed datasets, and their
performance metrics made publicly available online via an interactive
graphical front-end, and probably also via an API.

The main goal is an objective accuracy comparison of the major current
spike sorting codes. This should indicate for the electrophysiology community
the best code to use, and its expected accuracy (or distribution of
accuracies across units),
both of which may depend on context
(probe type, in/ex vivo, etc). A best code choice may also depend on
the available computing resources (RAM, GPU, etc).

A meta-goal is to gather information about which {\em quality metrics}, meaning
metrics computable without ground-truth data, most closely indicate ground-truth
accuracy when it is available.

Curating and gathering such datasets and metrics is a community effort.
Please give feedback and/or additions to this document.
There are three ways to comment (in decending order of convenience for us):
1) post a git issue to
this repository; 2) use the gitter chat room for this repository;
3) make a pull request with changes to this document.
\com{possibly via margin comments like this}


\subsection{Abbreviations}


\begin{tabular}{ll}
  GT & ground-truth(ed)\\
  ISI & inter-spike interval for one unit\\
AC & auto-correlation of firing events for a single sorted unit\\
CC & cross-correlation of firings between different sorted units\\
(BP)CM & (best-permuted) confusion matrix \cite{validspike}\\
e-phys & electrophysiological\\
\end{tabular}


\subsection{Mathematical symbols}
\label{s:symb}

\begin{tabular}{ll}
  $M$ & number of channels (electrodes)\\
  $N$ & number of time points (samples)\\
  $X$ & $M$-by-$N$ matrix of voltages at each time point, with elements $x_{mn}$\\
  $L$ & number of firing events output by a sorter\\
  $K$ & number of units found by a spike sorter\\
  $t_j$ & time of the $j$th firing event\\
  $k_j$ & label (ie, unit number or classification) of the $j$th firing event;
  $1\le k_j \le K$ for all $j$\\
  $m_j$ & peak electrode number (in the range $1$ to $M$) of the $j$th firing event\\
  $\tilde k$ & for a fixed GT unit, the best-matched sorted unit\\
  $Q_{kl}$ & confusion matrix element:
  number of time-matching events labeled $k$ in sorting 1 and $l$ in sorting 2\\
\end{tabular}


\section{Data and web interfaces}

One cannot proceed without a
standard {\em interface} between algorithms (codes) and e-phys datasets
(recorded inputs and sorted outputs).
Deciding on a commonly agreed-upon file format for e-phys datasets and the corresponding metadata is not an easy task.
However, this is simplified when we split it into two levels:
the ``mathematical'' interface
(ie the list of inputs and outputs), distinct from the
{\em implementation} of this interface (ie particular file formats).

We propose that the mathematical
interface for spike sorting be (see notation in Sec.~\ref{s:symb}):

\vspace{2ex}

\begin{minipage}[t]{1in}Inputs:\end{minipage} \begin{minipage}[t]{5in}
  $X$: Raw (preferably unfiltered)
  voltage recording, as a matrix of size $M$ (number of channels) by
  $N$ (number of time-points).
  In addition, the algorithm needs
  information on the sample rate (samples per second) and the 2D electrode coordinate locations. Other dataset-specific attributes such as the expected polarity of spikes (positive, negative, or both) may also be provided, which may or may not be used by the spike sorter.
  \end{minipage}

\begin{minipage}[t]{1in}Output:\end{minipage} \begin{minipage}[t]{5in}
  Lists of firing times $t_j$ and labels (unit or cluster assignments) $k_j$
  for each found event $j=1,\ldots,L$. Time is given in units of samples,
  ie numbers in the range $1$ to $N$.
  The labels are integers between $1$ and $K$.
  Events $j=1,\ldots,L$ need not be ordered in time, but should obviously
  be consistently ordered in the two lists.
\end{minipage}

\vspace{2ex}

{\bf Notes:}
\ben
\item
This minimal interface accommodates any kind of
algorithm, including those with very different internal mechanics such as ICA that do not have a detection stage \cite{validspike}.
\com{Benchmarking each stage, eg detection or snippet-based sorting, was discussed but is a separate task that we do not tackle here.}
All quantities of interest in a web-based exploration
of the sorting result---in particular mean or median spike waveforms---%
can be derived from the union of this input and output data.
Spike sorters are thus not
expected to output any additional data that could be reproduced from the interface data.
This ensures that the result summaries are algorithm-agnostic.
\item
  Probabilistic outputs were discussed at the meeting; however,
there are currently few if any spike sorters that produce such
outputs. We propose to postpone implementation of a probabilitic interface
until after the more basic system is in operation.
However, we note that the above interface could with minor tweaks
handle a sorter giving probabilistic outputs simply by
drawing $\sim 20$ independent samples from it and recomputing all metrics for
each sample, resulting in distributions of each metric. 
\een

\subsection{Formats for input dataset}
\label{s:int}

The raw voltages $X$ (multi-channel timeseries arrays) should be stored
in simply binary format,
with channels stored contiguously for each timepoint, ie column-major order
for the matrix $X$.
(Various formats were discussed at Janelia, but it was agreed that
avoiding time-consuming conversions between formats was essential.)
The accompanying metadata that the sorter has access to will
specify $M$ the number of channels, the sample rate (Hz), the duration $T$,
and the raw datatype (e.g. {\tt int16}, {\tt float32}, etc).
The metadata should
also contain a list of two-dimension coordinates of the electrodes,
preferably in $\mu$m units.
We are aware that some e-phys recordings are provided already filtered,
and thus it could be possible to include this filtering information in the
metadata that the sorter sees.

There are various strategies for encoding the above information
(JSON documents, plain text parameter files, python dictionaries, etc). We plan 
to write simple wrappers for converting between various systems used for encoding
the metadata. Thus we have the advantage of not requiring a single format at the
outset. Over time a particular preferred format may emerge, and it is also possible
that we will begin requiring a single format going forward. But we feel it is
premature to make such a constraint at this stage.

Finally, we may need to
accommodate the sitation where a large timeseries is stored in a sequence of
files to be concatenated.

\subsection{Format for sorted output}

We propose that codes output the above times and labels as the last two rows
of a single $3$-by-$L$ ``firings'' matrix.
The reason for the first row is that the sorter may optionally provide
``central'' electrode number $m_j$
(an integer between $1$ and $M$)
for each event $j$, and, if this is provided,
it is useful for the comparison framework.
Since, as discussed above, this central channel information can
also be extracted from $X$, $\{t_j\}$ and $\{k_j\}$, it is merely for optional
convenience. (Note that we do not require a strict definition of ``central'';
it could be the electrode where the peak amplitude of an event falls,
or of its corresponding mean waveform.)
If the first row contains zeros, our framework will replace it with
its own computation of $\{m_j\}$, using the peak amplitude of the mean
waveform for the corresponding label $k_j$.

To summarize, the firings output matrix contains:
\bi
\item row 1: optionally, the peak electrodes $m_j$, or a zero row.
\item row 2: the firing times $t_j$, in units of samples where
the first sample is $1$ and the last $N$. These should allow real numbers
(and should be double-precision since $N>10^8$ timepoints is common).
\item row 3: the labels $k_j$.
\ei
  
Since the middle row requires a 64-bit format, we propose that the
other two rows also be 64-bit double-precision, which will be rounded
to integers.
The file format for this matrix will either be MDA or HDF5.
For sorters that do not produce this output, we plan to
write simple output converter utilities.

\subsection{Hosting of ground-truthed datasets}

We already have in place a mechanism for uploading and sharing datasets.
We propose to host standard datasets, in two categories:
official datasets that metrics are computed on
(see the next section),
and unofficial test datasets that are needed for development and testing.
Each new candidate for a standard dataset should first be uploaded, viewed and tested before being included in the standard set.

\subsection{Hosting of sorting algorithms and outputs}

One discussion at the meeting was 
whether to upload {\em algorithms} (codes) or {\em sorted data};
the consensus at the meeting seemed to be both.
(Algorithms are harder to upload, but allow the website to show more complete
information such as runtime, RAM usage, etc; they also reduce the potential
for hand-tweaking of results.)

{\bf Algorithms.}
If a sorting algorithm is wrapped and registered as a MountainLab processor (see \url{https://github.com/flatironinstitute/mountainlab-js}), then the execution of that algorithm against the standard datasets may be automatically included in a nighly spike sorting run.
We have preliminarily wrapped several major algorithms this way, although further work is needed in this area.

All algorithms benchmarked have to be {\em fully automatic},
come with some {\em meta-data} such the parameter set, version number, etc.
They
should be cleanly separated from any visualization/curation/GUI tools.
Any curation or post-processing (eg based on their internal quality metrics)
should be automatic and included with the code.

All wrapped procedures would ideally operate on data in the standard agreed-upon format and output results in the standard format described above. However, we will also provide tools for converting between formats to accommodate different file conventions.

{\bf Sorting outputs.}
We also plan to allow
direct uploading of spike sorting results in the format described above.

Other ideas discussed at the meeting that we do not have plans to
implement yet include:
1) held-out or {\em hidden} data in the style of the NetFlix Challenge.
2) running codes from an arbitrary github repo with a dockerfile (which
will allow groups to automatically update their algorithms).





% DDDDDDDDDDDDDDDDDDDDDDDDDDDDDDDDDDDDDDDDDDDDDDDDDDDDDDDDDDDDDDDDDDDDDDDDDDDD
\section{Datasets}

It was agreed that a variety of brain regions, recording conditions, probes, and types of data are needed.
Here are some datasets with accompanying partial notes; most of these datasets
were mentioned at the meeting.
They fall into three categories.

\ben
\item {\bf Recordings with ground-truth.}
  
  Good features: gold-standard.
  Bad features: very small sample size,
  not in awake animals or various regions yet.
\footnote{It was noted that there are almost no ground-truthed recordings in
  awake, let alone behaving, animals. An exception is the recent
  thesis of Brian Allen \cite{allenthesis}
  in which the animal is only lightly anesthetized or awake.}

  \bi
\item Neto, Kampff et al. '16. \cite{neto16}
  32-channel and 128-channel (20 $\mu$m pitch)
  with single juxtacellular GT, in vivo rat,
  anesthetized, motor, sensory or parietal cortex, 30 kHz, roughly 10 min long.
  \url{http://www.kampff-lab.org/validating-electrodes/}
  
  We propose to include:

  \verb+2014-11-25_Pair 3.0+ : 32-channel, GT 52 $\mu$m from electrode,
  $N_\tbox{true} = 347$.

  \verb+2015_09_03_Cell.9.0+ : 128-channel, GT 29 $\mu$m from electrode
  (this is a bursting pair as discussed in \cite{mountainsort}),
  $N_\tbox{true} = 4895$.

  These appear to be the only datasets where the GT unit can be viably sorted.
  The GT units are very easy to sort, hence this might not be useful
  in differentiating algorithms.
  % The third to try would be 2014_03_26_Cell2.0 but I recall it being unsortable.
  
\item Yger et al. '18. \cite{yger18}
  252-channel ($16\times 16$ array, 30 $\mu$m),
  w/ loose patch, in vitro, mouse retina.
  Around 20 recordings of typical length 5 min, each with one GT unit
  with typically 500-5000 firings. Varying SNR of the GT units.
\com{The paper discusses at least 37 neurons; unsure how many GT are available.}
  
  \url{http://www.yger.net/software/ground-truth-recordings/}
% Password: gt252.
  
  Subset at \url{https://dio.org/10.5281/zenodo.1205233}

  To do: decide if all or a subset of datasets should be included.
  
  \item Franke et al. '15 \cite{frankeoverlap}.
    Tetrode with {\em dual} patch-clamp GT units, ex vivo rat cortex.
    33.3kHz.
    Includes 25000 spikes with overlaps of less than 1.5 ms induced by
    current injections, designed to test overlapping (colliding) spikes.
    To do: write to authors and ask for data.
    
  \item Bryan Allen \cite{allenthesis}
    and Caroline Moore-Kochlachs \cite{carolinethesis} dataset from Boyden lab.
    Dense array, $11$ $\mu$m pitch.

  \item Simultaneous calcium imaging and e-phys (Yuste and Shepard labs
    at Columbia; in progress; James Jun will have access to this).
    Provides low time accuracy in the spontaneous firing case,
    for approximately a dozen units in a single experiment.
    Optical excitation may remove time uncertainty.

\item
Brendon Watson's recent thread at 
\url{https://groups.google.com/forum/#!topic/klustaviewas/pfVC-CMSCTs}
mentions upcoming data from Dan English.
\ei
  
\item {\bf Hydrid datasets.}
  
  Good features: they embody the correct noise model.
  
  Bad features: biased towards already-sorted unit populations
  (which selects for, eg, high firing rate, manual selection)

  \bi
\item Steinmetz, Harris et al, 2016.
  \url{http://phy.cortexlab.net/data/sortingComparison/datasets/}
\item A hybrid dataset can be created by adding additional firings of
  sorted units to any dataset:
  this idea is also described as the ``spike addition'' stability metric in
  \cite{validspike}. This is similar to the Steinmetz/Pachitariu/Yger
  hybrid dataset construction, but differs in that the added firings are
  deliberately of the same type as found units.
  \ei
  
\item {\bf Recordings without ground truth.}

  Good features: abundant, possible to create partial GT by hiding a subset of electrodes.

  Bad features: no GT.
  
  \bi
\item Litke, Chichilnisky et al, 2004 \cite{litke}. 512-channel,
  hexagonal array, at least 2 hrs available.
  monkey retina, in vitro, 20 kHz.
  Used in YASS testing; have to check how much can be public.
  
\item Josh Siegle (Allen Institute), 2018. Six simultaneous neuropixels probes
  each with $\sim150$ channels within the brain. Length unknown.

\item Tetrodes and
  18-channel polymer probes from Jason Chung et al. \cite{mountainsort}.
  The tetrodes are hippocampal in behaving rats, and have three independent
  human sortings available as a vague GT.
  To do: check if data can be released.
  
\ei

  \item {\bf Simulated (in-silico) datasets.}
  
    Good features: abundant number of GT units,
    allows arbitrary electrode design and drift
    simulation.

  Bad features: unvalidated themselves,
  noise models too clean, no artifacts, no detailed modeling
  of electrode effects, harder to simulate non-rigid drift.

  \bi
\item BioNET (Allen Institute) simulations, by C. Mitelut.

  We suggest a single non-drifting and a single drifting dataset, with
  neuropixels probe geometry. Exist as 4-minute segments.
  
  These are expensive to run.
  
\item {\tt ViSAPy} simulations, Hagen et al. 2015 \cite{visapy}.

  The repo contains scripts for 16-channel polytrodes, etc.
  
  \url{https://github.com/espenhgn/ViSAPy}

  Yger et al. \cite{yger18} uses this code but doesn't report data.
  We have not yet used it.

  \ei
  
\een

Notes:
\ben
\item
Retinal data is quite different from cortical: retinal has
much lower noise, almost no drift, partial validation by planar coverage of
certain cell types (eg ON/OFF parasol), and axonal spikes that are non-local
(propagate across the entire array).
\item
We exclude other well-known datasets (eg Harris et al.
tetrode from 2000, Martinez et al. synthetic from 2009, Camunas-Mesa et
al Neurocube simulator from 2013) since they are either too small or have
been superceded.
\een


\section{Metrics}

We list metrics that could be reported for each algorithm-dataset pair.
Most are just sketched, pending formulae, then scripts that implement them.
A possible classification of metrics is as follows.
The three classes I, II, and III have descending order of
confidence. Thus class I most directly reflects the performance of the algorithm. Within a class no rank ordering is implied.
We first give an extensive list; below we propose a concrete subset.

\bi
\item {\bf Accuracy vs ground-truth} (Class I).

  We first choose $\tau \approx 1 $ms as the allowable spike time error;
  we have not found its choice makes much difference (timing only becomes an
  issue at the $\le 0.1$ ms level). However $2\tau$ should be less than
  the minimum true ISI; see below.
  
  We now consider a single GT unit. Let $n_\tbox{GT}$ indicate how many true firings it has, and $\{t_j^\tbox{GT}\}_{j=1}^{n_\tbox{GT}}$ be this set of firing times.
  Usually
  human work is needed to extract this set by manually thresholding eg an intracellular or juxtacellular recording (in our experience,
  thresholding can cause slight variation in number of GT firings claimed, due to bursting).

  For each sorted unit $k$,
  let $n_k := \#\{j:k_j = k\}$ be its number of events.
  Let $J_k := \{j:k_j=k\}$ be its index set
  in the firings array, with elements $j^{(k)}_i$ for $i$ in 1 to $n_k$.
  Let $t^{(k)}_i := t_{j^{(k)}_i}$, $i = 1,\ldots,n_k$ be its set of firings.
  The number of matches of unit $k$ to the GT unit firings is
  \be
  n_k^\tbox{match} := \#\{j : 1\le j\le n_\tbox{GT} \mbox{ and } \exists
  i \mbox{ such that } |t^{(k)}_i-t_j^\tbox{GT}|\le \tau\} ~.
  \label{nmatch}
  \ee
  This counts correctly only if the minimum GT
  ISI exceeds $2\tau$; this guarantees
  that each sorted event can be matched to at most one GT event.
  
  Selecting the best-sorted unit by the maximum matches has
  the problem (as Marius pointed out) that a sorted unit that fires continuously
  wins, even though it would have a ridiculous false-positive rate.
  Instead we define the best sorted unit $\tilde k$
  as the one having least overall inaccuracy, ie,
  \be
  \tilde k := \arg\min_{1\le k\le K} \eps^\tbox{inac}_k
  \label{tilk}
  \ee
  where $\eps^\tbox{inac}_k$ is defined by \eqref{inack} below.

  Note: even in the case of more than one GT units, we have decided to avoid optimizing for a bijective map from
  this set of GT units to the sorted units.
  This would require
  computation of a confusion matrix, the Hungarian algorithm,
  and with firing time restrictions this becomes algorithmically ill-defined
  \cite{validspike}.
  We believe it is not relevant unless there are a large number of poorly-sorted
  GT units to match.
  Thus it is possible that two GT units will be matched to the same sorted unit;
  we don't see a problem here since it could only happen in a poor sorting.

Finally come the Class I accuracy metrics:
  
  \ben
\item False negative fraction.
  For a given GT unit, and a given sorted unit $k$,
  this is the ratio of number of spikes missed to true number of spikes, ie,
  \be
  \eps^\tbox{FN}_k :=  \frac{n_k^\tbox{miss}}{n_\tbox{GT}}~,
  \label{FN}
  \ee
  where
  \be
  n_k^\tbox{miss} := n_\tbox{GT} - n_k^\tbox{match} ~, \quad k=1,\ldots,K~.
  \label{nmiss}
  \ee
  We report $\eps^\tbox{FN}_{\tilde k}$, ie, error fraction for the
  best matching sorted unit $\tilde k$.
  Is in range $[0,1]$, with smaller better.
  
\item False positive fraction. 
  For a given GT unit, and a given sorted unit $k$,
  this is the ratio of number of spikes unmatched to GT to
  number of sorted spikes, ie,
  \be
  \eps^\tbox{FP}_k :=  \frac{n_k - n_k^\tbox{match}}{n_k}~.
  \label{FP}
  \ee
  We report $\eps^\tbox{FP}_{\tilde k}$.
  Is in range $[0,1]$, with smaller better.

\item Overall error rate (inaccuracy).
  This combines the above: number of missed plus number
  of false positives, all divided by the union of true and sorted spikes,
  which simplifies thus,
  \be
  \eps^\tbox{FP}_k :=
  \frac{n_k^\tbox{miss} + (n_k - n_k^\tbox{match})}{n_k + n_\tbox{GT} - n_k^\tbox{match}} =
    1 - \frac{n_k^\tbox{match}}{n_k + n_\tbox{GT} - n_k^\tbox{match}}~.
  \label{inack}
  \ee
  This metric is used by James Jun \cite{jrclust}, and is
  similar but not identical to $1-f_k$ in \cite{validspike}.
  \com{A similar possibility is the mean of \eqref{FN} and \eqref{FP} for $\tilde k$.}
  Is in range $[0,1]$, with smaller better.

  \een
\item {\bf Biophysical metrics} (Class II).
  \ben
\item rate of refractory violations (ISI, or AC plots),
  separately for each unit. Requires choice of ISI lower cut-off, eg 2 ms.
  Called $\mbf{f}_1^p$ in \cite{Hill2011}.
\item Existence of unphysical notch in CC. This is useful
  for seeing missed spikes due to collisions; see eg \cite{chaitu}.
  We need to choose a notch width. This could be computed faster than
  the entire set of CCs. Related to $\mbf{f}_3^n$ of \cite{Hill2011}.
\item Highly-one-sided CC, indicating a bursting pair (or triplet, etc),
  that has been mistakenly split. Parameters need to be fixed here.
  Of course, one-sided CCs can occur legitimately due to synaptic coupling;
  we seek expertise on this.
\item ``Burstiness'', assessed by off-diagonal return map
  (scatterplot of successive ISI for a putative unit, as in JRclust).
\item Refractory gap in CC between a pair that matches the AC of each in the pair; this indicates a false split, due to eg drift.
\item Disappearance of a unit over time, as an indicator of failure
  to handle drift.
\item
  Validation by spatial localization of place cell
  firings \cite{mountainsort} (special to hippocampus of behaving rodent).
\een
  
\item {\bf Quality metrics} not requiring ground-truth (Class III).
  
  These are ``surrogate'' metrics for per-unit quality that
  can be quoted after a sorting.
  We don't yet know which are the best indicators of accuracy.
  Crucially, they can (and must) all be able to be
  computed from only the input and output data of Section \eqref{s:int}
  ($X$, $\{t_j\}$ and $\{k_j\}$),
  and are independent of the sorting algorithm.
  \ben
\item Peak amplitude of mean template, divided by RMS noise level,
  after filtering.
  Ie, peak SNR. This is the max over time and electrode number.
  Noise level ($\sigma$)
  may be robustly estimated using median absolute deviation (MAD) computed
  from filtered data,
  scaled to a Gaussian equivalent $\sigma$ \cite[(3.1)]{spc}.
  This requires a standard definition of filter parameters purely for the
  computation of this quality metric.
  (We would exclude spatial whitening from this.)
\item Noise-overlap \cite{mountainsort}. Is cluster-shape agnostic.
  \footnote{This replaced the idea of amplitude histogram separation from detection threshold.}
\item Isolation \cite{mountainsort}. Is cluster-shape agnostic.
\item 1D projections (eg, amplitude histograms) as used in Pachitariu et al \cite{kilosort}
\item Mahalanobis-style estimates of false-pos and false-neg rates
  based on a Gaussian clusters assumption. Requires (re)computing a local
  feature space.
  \item Various other quality metrics from Hill et al \cite{Hill2011}
  \item Various stability metrics requiring reruns of the sorter \cite{validspike}.
\item Community-supplied quality metrics, that could be uploaded automatically as scripts acting on the data\ldots
  \een

\item {\bf Other metrics} and meta-data (Class IV)
  \ben
\item Algorithm run-time in seconds. Runs our framework performs would
  all be on the same machine.
  This may need a CPU-only and CPU+GPU category.
\item Peak RAM usage.
\item Peak disk usage (temporary intermediate files).
\item Subjective user opinions on ease of installation, use, parameter adjustment effort, and scalability for large datasets.
  \een
  
\ei

Since algorithm comparison is also needed, Class V could be metrics
of similarity of two sorting outputs:
this we propose to be based on the {\em best-permuted confusion matrix} (BPCM)
as described in \cite{validspike,mountainsort}, which is
diagonal if two sortings match.
The website should be able to produce and display graphically
the BPCM between any two
algorithms run on the same data.

\subsection{Initial implementation plan}

In the first released website we expect to implement from the above list the following.
Class I: 1,2,3.
Class II: 1.
Class III: 1,2,3.
Class IV: 1.

%To be debated.



\section{Prior work and influences}

We are influenced by several previous versions of online
comparison tools, including:
\bi
\item \url{http://neurofinder.codeneuro.org/}\\
  J. Freeman's calcium-imaging algorithm comparison.
  Intuitive interface (click and mouse-over for more detail), submission info,
  gitter chatroom, contest, cash prizes.
  We may or may not want the competitive aspect of leaderboard style.
  No data or outputs are visible, just scores.
\item \url{http://spikefinder.codeneuro.org/}\\
  P. Berens, spikes from calcium fluorescence curves, similar to above.
\item \url{http://spike.g-node.org/}\\
  In this now-defunct
  2011-2012 project the user
  uploads sorted data, which is compared against a {\em hidden} ground truth sorting and optionally published. Layout is a little non-obvious.
  Tags and owners of algorithms are good.
\item \url{http://phy.cortexlab.net/data/sortingComparison/}\\
  N. Steinmetz comparison of several algorithms on hybrid data.
\item \url{http://www.spikesortingtest.com/}\\
  C. Mitelut's comparison site. Good collection of recent
  simulated data, but
  no algorithms, not yet used by others. Metrics: purity and completeness.
\item \url{http://simonster.github.io/SpikeSortingSoftware/}\\
  Large but incomplete list of codes and their features.
\ei

\section{Graphical web interface}

To do: describe web interface and which summary plots can be brought up.


\section{To do}

Decide whether filtered data is accessible, or if filtering is always
taken to be internal to an algorithm.

Write formulae for initial Class II and III metrics.

Write reference scripts to compute the initial quality metrics from only
$X$, $\{t_j\}$, $\{k_j\}$.

Go through datasets and be more specific about which to include, and their
length and size (GB).

Ask re Franke data.

Quantify and collect the simulated datasets.


\bibliographystyle{abbrv}
\bibliography{spikesorting}

\end{document}
